% !TEX root = all.tex
We reviewed some of the most canonical parameters for thermal comfort in this paper, focusing particularly on their limitations and underlying simplifications. Our hypothesis of the occupants as larger groups went through a two-step simplification process was verified, upon which we also identified some of the most recent efforts to address the thermal comfort gap - which we believe is caused by the negligence of both the individual differences between occupants and the human physiological responses.

We discovered many problems in the existing simplification methods from a group of occupants to a single hypothetical occupant - it is not to say that the occupants cannot/should not be simplified, but the context of this simplification needs to be better undrstood among researchers and practitioners.

We also feel we need to better undrestand what and how many of thosesimplifications are used when their resuls are cited and used elsewhere. Many of the assumptions were passed on alongside their cited results. This also affects the data-drive approach when reviewing these papers, since the underlying assumptions will again go unexamined, and the asumptions get passed on without being re-examined.

Using direct methods to collect the outputs from occupants may be an alternative that could solve the comfort gap. Existing research in this direction uses either direct physiological measurements or actual feedback from the occupants, which can both be viewed as attempts to re-introduce the occupants back to the control loop of building systems. Other efforts includes the status-quo research on personal thermal comfort models that characterizes the individual thermal differences, which includes both the physiological feedback and the actual votes from the occupants. We believe personalized thermal comfort model is a promising alternative to the existing comfort/occupent-centric explorations, despite their limited scale at the time of this paper. 


To re-introduce the occupants to monitoring and control of indoor environment, it is necessary to measure their responses directly. With the recent development in the non-intrusive measurement IoT and wearable sesors, measuring the physiological signals from the occupants directly have much easier. The challenge remains, however, to discren the individual discrepancies between the occupants - how their sex, age, build and corresponding metabolic rate may differ to properly characterize their personal comfort. Collecting subjective feedback could be a very helpful tool to help bypass these individual differences without collecting personal identifiers. However, it is crucial for reserachers to clearly discern thermal sensation, thermal preference and thermal satisfaction when designing experiments since each individual concepts overlap but are fundamentally different. 

We wanted to focus primarily on the two step abstraction of the thermal comfort of occupants in this paper and concentrated on how different simplifications and assumptions chronologically emerged during this simplification. This is further coupled with investigation on how recent awareness of the importance of thermal comfort. Whata re some of the basic traits of C++ as a language? How is this practical - or non-practical is a key problem that we're currently facing. 
