% !TEX root = all.tex

The first and foremost simplification that we observe in the existing models of thermal comfort is how a group of occupants became a single occupant, or a synthetical one for that matter.

Fanger's research back in the 1970s were among the first to suggest that occupants can be represented by a single hypothetical occupant\cite{fanger_thermal_1970}. Based on his own deterministic thermoregulation-based model that calculates the absolute state of thermal comfort devleoped in 1967\cite{fanger_calculation_1967}, his dissertation asserts a calculated PMV within the range of -0.5 and +0.5 is sufficient to be representative of the entire population of any occupant group. Using experimental results collected from 1,394 college students, Fanger concluded that demographic variations among the occupants are not significant enough to affect the thermal comfort conditions. Although there are many further research suggesting otherwise\cite{kingma_energy_2015,wang_individual_2018}, the simplification of the occupants into a single occupant gradually became mainstream\cite{ansi/ashrae_standard_2017}.


\subsubsection{Outputs from the occupants}
        Physiological responses - why aren't they mesured?
        State of art
        Challenges in directly measuring physiological feedback of the human body
        Another form of output from the occupants, the physiological signals from the occupants 
        Challenges for Using actual comfort votes as control feedback signal

\subsubsection{Inputs from the occupants}
	Inputs from 
	The first and perhaps the largest simplification among occupants is the metabolic rate. Following Fanger's proposition in 1967 \cite{fanger_calculation_1967}, the metabolic rate of the occupants has been assumed to be 58.2$W/m^2$, or 1 MET in most of the follow-up literatures and regulations, such as the ASHRAE Standard 55\cite{ashrae_ansi/ashrae_2013} and ISO 7730\cite{iso_iso_2005}. 
        Metabolic rate, clothing factor, age, build and sex are all ignored in conventional comfort modelling, where a middle-aged, medium-built hypothetical man is considered.
