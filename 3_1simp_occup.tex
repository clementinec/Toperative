% !TEX root = all.tex

The first and foremost simplification that we observe in the existing models of thermal comfort is how a group of occupants became a single occupant, or a synthetical one for that matter.

Fanger's research back in the 1970s were among the first to suggest that occupants can be represented by a single hypothetical occupant\cite{fanger_thermal_1970}. Based on his own deterministic thermoregulation-based model that calculates the absolute state of thermal comfort devleoped in 1967\cite{fanger_calculation_1967}, his dissertation asserts a calculated PMV within the range of -0.5 and +0.5 is sufficient to be representative of the entire population of any occupant group. Using experimental results collected from 1,394 college students, Fanger concluded that demographic variations among the occupants are not significant enough to affect the thermal comfort conditions. Although there are many further research suggesting otherwise\cite{kingma_energy_2015,wang_individual_2018}, the simplification of the occupants into a single occupant gradually became mainstream\cite{ansi/ashrae_standard_2017}.


\subsubsection{Physiological Feedback - Outputs from the occupants}
	Most of the popular thermal comfort metrics relies on thermoregulation models of the human body - which is a subject well-studied prior to the proposition of the PMV/PPD model\cite{fanger_thermal_1972}. From the physics of the microclimate, measurement and models of the human heat balance and their relationship to thermal comfort, research that supports proper characterization of the human body were widely available as early as the 1980s, and were systematically organized into series of publications(Cena, 1981). 

	Actual implementation of these models turned out to be much more complicated than what the literature outlined. Nishi outlined the heat balance of the human body needs to have the radiation, convection, evaporation and conduction between the skin surface with the thermal environment. To do so, Nishi claimed that it is necessary to monitor physiological variables including the skin temperature, skin wettedness, mean body temperature, metabolic energy consumption and the rate of external work(Gagge and Nishi, 1977). Together with the air temperature and water vapor pressure, Gagge and Nishi demonstrated that these seven input parameters are necessary to model the heat balance of the occupants. Unlike the environmental parameters, the physiological parameters are much more challenging to measure and monitor.

	% Skin temperature
	Skin temperature of the human body varies significantly across different parts of the body \cite{choi_cobi:_2010}, and has been correlated with triggering different thermal responses in some of the latest studies. Its measurement has also transformed from attempting to capture the full temperature distribution through thermography (Clark, mullan and pugh, 1977) to strapping skin temperature sensor to the wrist of the occupants \cite{choi_investigation_2017,liu_personal_2019} as well as direct indicator of occupant thermal comfort. However, the wrist temperature sensors could have limited accuracies \cite{mccarthy_validation_2016} and will require prolonged calibration among larger groups of occupants. This has not hindered some more recent studies of creating prototypes that takes multiple skin temperature readings on the wrist. Beginning with single-point measurement of skin temperatures on the upper-extremity \cite{wang_observations_2007}, the state-of-art measurement have expanded to include more points of data collection. Sim et al., for example, developed a wrist band that contains four temperature sensors, measuring the radial artery, ulnar artery, upper wrista dn the fingertip temperature, which appeared to be better correlated to the thermal sensation of the occupants. Other studies have also incorporated the thermal imagery of the occupants under investigation, but not as the primary but rather a complimentary research method alongside air temperature and relative humdity measurements\cite{lu_thermal_2019}. 

	Other metrics that can be measured as an `output' from the occupants during an evaluation period of the occupants' thermal comfort include the mean body temperature and the core temperature of the occupants. The mean body temperature, similar to the core temperature of the human body, is often considere to fluctuate within a much smaller range (36.5 - 37.5 $\degree C$, \cite{shapiro_environmental_1984})when compared to skin temperatures. To directly measure the core temperature of the participating occupants, ingestible thermometer pills are often used (Huizenga, 2004). This will likely not be non-invasive and cannot be used for proplonged period of time or used during the real-time operation of the buildings due to the nature of the sensing mechanism. However, since the flucuation of the core temperature remains small, it is not uncommon to assume that the core temperature remains relatively constant during the operation of building systems\cite{gagge_effective_1971}. Studies that attempted to establish links between the core temperature and the diurnal outdoor air temperature have pointed to a possible correlation (kakitsuba,2019), which is consistent of ealier findings from Doherty and Arens\cite{doherty_evaluation_1988}.

	Skin wettedness is much less investigated among the existing models. Per its definition, skin wettedness is the proportion of the total skin surface area of the body covered with sweat. Skin wettedness is, therefore, not a parameter that can be directly measured, but rather often approximated as with the clothing factor\cite{mcintyre_subjective_1972}. While most research regards the skin wettedness as a process variable that can be calculated from water vapor pressure\cite{doherty_evaluation_1988}, some other research also linked the skin wettedness to clothing factor, i.e. higher clothing level could induce higher vapor resistance and therefore higher skin wettedness\cite{havenith_personal_2002}. There has yet to be much investigatio on how different clothing values, specifically the transient clothing values when taking into the layering-up or down of the occupants into the thermal comfort evaluations of the occupants. And fundamentally, as clothing factor is comparatively more loosely defined a concept compared to other measurable `outputs' of the occupants, it could be challenging to identify its real-time values on the occupants without using machine-learning or deep-learning on the live feed of the occupants, which may raise privacy concerns. 

	In addition, there are also other `outputs' from the occupants that are often considered constant in exisiting metrics of thermal comfort. The rate of doing external work, for example, is often considered to be close to or equal zero\cite{fanger_thermal_1972}. This is both due to the assumption that office workers are only doing light office work which does not require excessive physical strain that drastically alter their physiological conditions. Comapred to the amount of physical work that an office worker is conducting, it is far much more common to estimate the level of activity among existing literature, albeit their respective limitations. 

	The importance of characterizing the activity level of occupants within the indoor environment, although directly associated with the metabolic rate of the occupants, can often be found discussed independently. Akimoto et al. investigated the thermal comfort conditions of workers in an office environment and found significant variations of activity levels among the occupants\cite{akimoto_thermal_2010}. However, most of the existing studies base their activity estimation on occupants' self-report \cite{huang_potential_2015} or measured accelerations. Accelerometers can help establish the activity levels of the occupants\cite{rothney_validity_2008}, but are often associated with measuring the correct level of activity rather than associating activity levels to the resulting thermal comfort (or sensation)of the occupants. Mishra et al. went a step further and termed the measurement of activity as `actimetry' where the measurement of skin temperatures is coupled with the measurement of an accelerometer embedded in an armband worn by the occupants\cite{mishra_actimetry_2019}. This accelerometry technique has been previously shown useful in quantifying the rate of energy expenditure during activity for animals \cite{wilson_moving_2006} and therefore may also be useful for in attempting to quantify the energy spenditure of the occupants. 
	To this end, the understanding of characterizing activity level remains, not much beyond how Gagge et al. described the relationship between the physiology and thermoregulatory system and the thermal comfort\cite{gagge_comfort_1967}. 

	Researchers have also attempted to use other `outputs' from the occupants to correlate to their state of thermal comfort. Liu et al. investigated whether recording electrocardiogram (ECG) of the occupants can reflect their thermal sensation \cite{liu_heart_2008}, and was successful in proving the usefulness of low- and high-frequency bands (LF/HR) can be used to estimate the occupants' thermal sesations. Follow-up studies have shown the limitations of heart rate monitors \cite{gillinov_variable_2017}, which may prohibit their further applications in real-time measurement of thermal comfort or sensations. Nkurikiyeyezu et al. built on these research and suggested to use the variability of heart rate (HRV) instead of the heart rate themselves as an indicator of thermal comfort, which they demonstrated to have prediction accuracy of up to 93.7\%. This study did not clarify on the ECG monitoring method used by the researchers and landed on a forward-looking note towards development in IoT sensors and wearable technologies, which falls back to the limitations as pointed out by Gillinov et al.\cite{gillinov_variable_2017}. Using heart rate sensors to provide accurate and reliable feedback of the activity levels and heart rate sensors appears to require additional effort in the foreseeable future. 
	Recognizing the challenges posed by wearable sensors and the discomfort they may create, Anguita et al. proposed a smartphone-based technique where they exploited the accelerometer on the smartphone and used it to collect activity data which were then used to train a Support Vector Machine (SVM) model. Among many of the publications in this domain known as Human Activity Recognition (HAR), the goal of this study is to understand and characterize the Activity of Daily Living (ADL) of the occupants instead of providing inputs to thermal comfort - which does suggest their valuable nature when deployed within the building sector. 

\subsubsection{Individual differences - Inputs from the occupants}
	Regarding the modeling of the actual occupants in an indoor environment, it is often necessary to go through a two-stage abstraction of the occupants, wherein the occupants first became a hypothetical and synthetic person, and then becomes a specific environmental parameter.  as researchers claim the male/female or age-based variations of thermal comfort is relatively small. However, since the existing assumption of the hypothetical occupant remains to be a middle-aged man with a specific built and height, variations of metabolic rates alone cannot satisfyingly characterize the existing occupant profile of a given environment, let alone the different levels of clothing factors, age, build and their corresponding effect in the thermal comfort of the occupants. And since they all inadvertently tie into the thermal comfort model through either metabolic rate or clothing factor, they can all be considered inputs from the occupants that represent individual differences - and are often overlooked in the existing models. 

	The first and perhaps the largest simplification among occupants is the metabolic rate. Following Fanger's proposition in 1967 \cite{fanger_calculation_1967}, the metabolic rate of the occupants has been assumed to be 58.2$W/m^2$, or 1 MET in most of the follow-up literatures and regulations, such as the ASHRAE Standard 55\cite{ashrae_ansi/ashrae_2013} and ISO 7730\cite{iso_iso_2005}. 
    The metabolic rate split is one of the most studied parameter that can result in variations between predicted thermal comfort and actual thermal sensation. Many researchers have associated this to metabolic rates, since different levels of metabolic rates can cause estimations can lead to over-estimation of the cooling demand during sizing of air handling systems. Despite being sometimes categorized into how important it is to categorize the different levels of activity within an indoor environment\cite{rupp_associations_2018}, researchers have continuously reported the need to characerize individuals differently in environments such as hospitals where the activity levels vary significantly enough to create a large differentiation between the preferred thermal condition between carers and patients (Hill, Care, 2000). 

    One of the most investigated theme in indivitual thermal comfort variations is the male/female seaparation, where the female occupants are found to be more commonly dissatisfied with standard-manded thermal environments\cite{kingma_energy_2015}. Kingma and van Marken Lichtenbelt linked this to the energy system requirements, arguing energy savings may be achieved when the occupants' thermal requirements are better understood - i.e. female are no longer under-represented among the occupants\cite{kingma_energy_2015}. An earlier study from Doherty and Arens pointed out in 1988 that the gender differences were found to impact the accuracy of their thermal comfort prediction model, but lacks supporting evidence since the sample size was too small\cite{doherty_evaluation_1988}. Existing research also points to different levels of satisfaction among male and female occupants. Karjalainen concluded that female occupants are in general less satisfied within the thermal environment as the male occupants \cite{karjalainen_thermal_2012}. Significant variations between the male and female occupants with respect to their thermal comfort, thermal sensation and thermal perception were found in more recent studies by Maykot et al. where they compared and highlighted the differences between the different concepts as well as the male/female differences. It is, however, worth noting that the amount of votes of the female occupants are 40\% lower than that of the male occupants, which may cast doubts on the validity of their conclusions.



