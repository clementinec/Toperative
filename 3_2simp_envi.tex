% !TEX root = all.tex    
    Defined as the temperature of a homogeneous sphere that exchanges the same amount of heat as the actual surrounding with the human body, mean radiant temperature is arguably the most problematic environmental parameter within both the indoor and outdoor environment \cite{kantor_most_2011}. Its definition is very easy to follow but difficult to compute, particularly due to the challenges of quantifying the view factors betwen the human body and the surrounding environments. In the meantime, mean radiant temperatures are required to calculate thermal comfort and heat stress indices such as the UTCI(Jendritzky, 2012), PT (Staiger, 2012), PET (Hoppe,1999) as well as PMV \cite{fanger_calculation_1967}. 

    Recent expansions of its definition to include the influence of the sun also add up to the challenge, as the mean radiant temperature now has two subsets with respect to the wavelength of the incoming radiation: longwave and shortwave\cite{ansi/ashrae_standard_2017}. Partially due to these inherent complexity of the mean radiant temperature, it is one of the most abstracted environmental parameter of all the environmental parameters. These simplifications are shared among existing standards (such as ISO 7726 \cite{standardization_iso7726_2001} and ASHRAE Standard 55 \cite{ansi/ashrae_standard_2017}) with both literal definitions and measurement techniques that are underlined with various simplifications. 
    \subsubsection{One MRT in one room}
    %Standards - 7726 and 55
    The first and potentially the most well-accepted simplification is mean radiant temperatures are singular to every single room. This is a very common simplification among the existing literature and standards from both ISO (ISO 7726\cite{standardization_iso7726_2001}, ISO 7730\cite{iso_iso_2005}) and ASHRAE (Standard 55 \cite{ansi/ashrae_standard_2017}). Although both standards acknowledges the importance of measuring MRT at multiple locations when there are radiant asymmetries present, the overarching understanding across them was that for indoor environments, a single point inside a room is enough for MRT evaluation. 
    Within the ISO Standard, the thermal comfort ergonomics and its measurement was meant to be the focus. Despite acknowledging the  Standard 55 - 2017 from ASHRAE provided similar guidelines where the literal definitions of mean radiant temperature and the relationship between the long wave and shortwave radiations. 

    The overlooked spatial variation of MRT remained a topic of interest for some researchers. Earlies efforts on characterizing and quantifying thermal comfort alongside Fanger's included some significant explorations on how the MRTs relate to the thermal comfort of the occupants where different mean radiant temperatures were measured at multiple locations. An interesting follow-up for this radiant connection to thermal comfort came from DeGreef et al. where the spatial MRTs were simulated and ploted as a contour map\cite{degreef_simplified_1998}. Very few studies followed this example. 

    %Device-specific? Another layer of how MRT is understood.
    Aside from the influence of the existing regulations and standards, the obscure defintion of mean radiant temperatuer have also led to a significant level of confusion about how it should vary through space. Within the existing standards, the literal definition of mean radiant temperature remains consistent, where it is the temperature of the homogeneous sphere surrounding a person exchanging the same amount of radiant heat as the actual surrounding. For the purpose of clarifying the role of surrounding surfaces, the mean radiant temperature can be expressed with Equation~\ref{eq:trsurf}, where the mean radiant temperature is the sum of the surrounding temperatures weighted by their corresponding view factors. However, due to the wide usage of globe thermometers, it is not uncommon to refer to Equation~\ref{eq:tg} also as an expression of mean radiant temperature\cite{graves_globe_1974}. 
    When referring to the measurement of the mean radiant temperature, ISO 7726 clearly outlines that instruments such as globe thermometers and net radiometers are used in `deriving' or `approximating' the MRT values. However, it is not uncommon for textbooks to introduce these means of alternatives as alternative explanations of the concept. And since the definition of MRT can be challenging to visualize to begin with, mean radiant temperature and globe temperature gradually becomes interchangeable \cite{kantor_most_2011}. Because of its affordability and simplicity to assemble, globe thermometer has became not only a popular apparatus of measuring MRT within the indoor environment, but also in urban climate studies \cite{thorsson_different_2007}. 
    \begin{equation}
    T_r = \sum_i^N T_i F_{p-i} \label{eq:trsurf}
    \end{equation}

    \begin{equation}
    T_r = [T_g + 2.5\times 10^8 \cdot v_a^{0.6}(T_g-T_a)]^{1/4}\label{eq:tg}
    \end{equation}

    Problems in under MRT as a concept: difficulty to interpret Iterations attempting to simplify MRT: ISO, surface averaged, etc. Causes problems in measuring MRTs. 

    %Fundamental difficulties of calclating the view factors? Direction of the human body? clothing surfaces, emissivity, etc.?
    Fundamentally, the calculation of the respective view factors for different surrounding temperatures can be difficult and challenging. Fanger approached the problem in his seminal publication on thermal comfort \cite{fanger_thermal_1970} by creating reference curves with different ratio between the occupant and the targeted surface. This has encouraged some follow-up research where the corresponding surfaces of a room are segmented into smaller rectangles and have their corresponding view factors calculated (Chapman \& Zhang, 1996). DeGreef and Chapman followed up on this and improved the method to calculate the view factors from triangles while computing their surface normals, and was able to produce a more comprehensive set of view factors for characterizing MRT distributions \cite{degreef_calculation_2017}. 

    %Simulation methods when obtaining MRT(CBE MRT tool)
    Existing tools that evaluates the spatial variation of MRTs are also very rare. On the simulation side, the most common building simulation engines such as EnergyPlus \cite{energyplus_engineering_2013} do account for the temperature variations and calculate the respective view factors, but still remains to assess the mean radiant temperature as a single node inside a specific thermal zone despite accounting for the surrounding surface temperatures and their respective view factors. More recently, the Center for the Built Environment published a Spatial Thermal Comfort Tool \cite{arens_modeling_2015} that focuses on the spatial resolution of mean radiant temperature, which calculates both the spatial MRT and the corresponding PMV assuming constant inputs from the other five parameters ($T_a, v_a, RH, clo, M$).  

    For researchers focusing on the outdoor environment, the single-valued MRTs are much less common, particularly due to the influence of fluctuation of shortwave solar radiation\cite{middel_sky_2017}. And as most models requiring MRT as inputs to yield meaningful outdoor thermal comfort or heat stress results, measurements for mean radiant temperatures are often intentially conducted at different locations\cite{thorsson_different_2007,thorsson_thermal_2007}. There has also been many recent development on quantifying both the shortwave and longwave radiation in the most recent studies. Arens et al. studied the effect of solar (shortwave) radiation's effect on the indoor thermal environment \cite{arens_modeling_2015-1}, and was followed by Marino et al. to characterize the reflection of the shortwave inside an indoor environment\cite{marino_effect_2017}.      
    % The relationship between MRTs and the resulting PMVs have also been a subject of interests for many studies. According to 
\subsubsection{Air temperature and MRT}
        Aside from assuming that mean radiant temperatures can be treated as a homogeneous parameter for an indoor environment, another very common simplification is to consider mean radiant temperature the equivalent of air temperature\cite{kantor_most_2011}(Langer, 2013; matzarakis 2008). 
        
        It is difficult to estimate how much of this simplification is driven by the ASHRAE and ISO standards, both of which pointed out that the mean radiant temperature can be simplified into the measured air temperature for a homogeneous air-conditioned indoor environment. It is nearly as difficult to attribute all of those claims to the use of globe thermometers, who remains to be one of the most popular apparatus when measuring mean radiant temperature\cite{kantor_most_2011}. It is, however, to identify a few key publications that specifically studied this problem and pointed out the fundamental differences between the air temperature and mean radiant temperature - and how replacing the mean radiant temperature with air temperature can be problematic. 

        Chaudhuri specifically analyzed this assumption in 2016 with an experimental study, where they found MRTs to have the highest positive correlation with occupant-reported thermal sensations, and the simplification affects the discomfort even more within the uncomfortable ranges \cite{chaudhuri_assuming_2016}.

        Walikewitz conducted another case study a year before where they found mostly negligible differences between air and mean radiant temperatures throughout their experiment. An interesting highlight of this particular study is the effect of solar radiation on the resulting mean radiant temperature - which can in turn affect the resulting thermal comfort significantly \cite{walikewitz_difference_2015}. 

        And fundamentally, the measurement of mean radiant temperature can also be problematic. Thorsson et al. conducted an analysis on methods used to characterize mean radiant temperature where the expression of convection heat loss of the globe thermometer is carefully calibrated\cite{thorsson_different_2007}. The expression of the mean radiant temperature for globe thermometers used in ASHRAE Handbook \cite{american_society_of_heating_2013_2013} and the ISO standard both pointed out the need to measure air velocity, but does not specify the precision or accuracy necessary for the measurement, which may sifnicantly affec the resulting mean radiant temperature. In addition, the size, material and inherent limitation of using a globe thermometer all remained un-addressed in the current literature. According to an early report in 1974, globe thermometers are more accurate when their settling time is long enough to not be scrambled by localized air flow but not too long to capture temporal changes within the indoor environment \cite{graves_globe_1974}. Without precise characterization of the convective heat transfer between the globe thermometer and the surrounding air, the resulting mean radiant temperature obtained through globe thermometers remains much more dependent on air temperature than the radiant environment, and hence the underlying explanation on how air temperature can be considered consistently a satisfying proxy of mean radiant temperature. 
 

        More recent investigations have pointed to a clearer limitation of the globe thermometer, while the measurement of mean radiant temperature within an indoor environment remains to be acceptable at the center of the room \cite{dambrosio_alfano_measurement_2013} so long as there are no radiant asymmetries. 

        

            Scenarios where Air IS NOT MRT needs to be better recognized:
            Radiant systems
            Shortwave radiation through huge fenestration systems
            Larger view-factors of adjacent cold/hot surface areas: ocean/river
\subsubsection{Simplified RH}
        Maybe briefly mention how we are suggesting we have already kept HR in check?
        Complexity in creating two-objective systems? Price?
% \subsection{Other sensed data}
%     A good(?) question is whether we want to keep the structure of the paper - or add more stuffs. There's the performance gap hat should go in somewhere, and problem of the two-stage simplification is that it's not super clear yet. How do you simplify a group into a person, that part is not clear enough. Or is it? In 3-1 and 3-2?

\subsection{Review of Recent Literature}
    To better understand how are these simplifications and assumptions affecting the status-quo research, we have also conducted a data-driven literature review, where we focus on finding the actual metrics measured in thermal comfort studies conducted since 2000 to 2020, as was identified by Park \& Nagy \cite{park_comprehensive_2018} to be relevant period of thermal comfort research. The challenge of manually reviewing the thermal comfort research that has been published over the last 20 years, we leveraged the free academic search engine demensions.ai to collect the bibliographical data including keywords, abstracts and citations and export them to be analyzed with TextBlob with Python after initial processing in VOSViewer. We selected dimensions.ai over other portals since it has a few key strengths over the more common databases like Google Scholar and Web of Science, as it contains not only more expanded journal publication records, but also grants, patents and policy documents. Using its dedicated API, more explicit requests for publication records are also possible. 

    %Method, WoS, Journal, data collection
    To collect the publication records for the purpose of this study, we used the search term \textit{"thermal comfort"} and \textit{"buildings"} to identify the whole pool of relevant research published between 2000 and 2020. The classification terms for the full literature dataset are respectively the six components contributing to thermal comfort are then used respectively \textit{"air temperature"}, \textit{"mean radiant temperature"}, \textit{"air velocity"}, \textit{"relative humidity"}, \textit{"clothing factor"} or \textit{"metabolic rate"} - as identified by Fanger in 1967\cite{fanger_calculation_1967}. 

    As a result, we identified 44,532 journal publications from the Dimensions engine of papers published between 2000 and 2010. We downloaded the metadata of the publication, including the title, abstract, author, citation, publicatin year for further analysis. This was collected between 586 journals, among which the most cited remains to be \textit{Building and Environment} and \textit{Energy and Buildings}, which was consistent with Park and Nagy's finding\cite{park_comprehensive_2018}. 