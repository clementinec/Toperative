% !TEX root = main.tex
\subsubsection{Mean Radiant Temperature}
    Defined as the temperature of a homogeneous sphere that exchanges the same amount of heat as the actual surrounding with the human body, mean radiant temperature is arguably the most problematic environmental parameter within both the indoor and outdoor environment \cite{kantor_most_2011}. Its definition is very easy to follow but difficult to compute, particularly due to the challenges of quantifying the view factors betwen the human body and the surrounding environments.
    Recent expansions of its definition to include the influence of the sun also add up to the challenge, as the mean radiant temperature now has two subsets with respect to the wavelength of the incoming radiation: longwave and shortwave\cite{ansi/ashrae_standard_2017}. Partially due to these inherent complexity of the mean radiant temperature, it is one of the most abstracted environmental parameter of all the environmental parameters.

    The first and probably foremost simplification for the mean radiant temperature is that mean radiant temperature is a room-specific variable. This simplification is very wide-spread among the existing literature and standards such as the ISO 7726\cite{standardization_iso7726_2001}, where the ergonomics of the thermal environments and how it should be measured is the main focus.  

            Problems in under MRT as a concept: difficulty to interpret Iterations attempting to simplify MRT: ISO, surface averaged, etc. Causes problems in measuring MRTs.
\subsubsection{One MRT in one room}
            Measurement protocols of MRT and IEQ (check PoE protocols).
            Simulation methods when obtaining MRT
\subsubsection{Air == MRT}
            Cases where this assumption might be true.
            Scenarios where Air IS NOT MRT needs to be better recognized:
            Radiant systems
            Shortwave radiation through huge fenestration systems
            Larger view-factors of adjacent cold/hot surface areas: ocean/river
\subsubsection{Simplified RH}
            Maybe briefly mention how we are suggesting we have already kept HR in check?
            Complexity in creating two-objective systems? Price?
\subsection{Other sensed data}
    A good(?) question is whether we want to keep the structure of the paper - or add more stuffs. There's the performance gap hat should go in somewhere, and problem of the two-stage simplification is that it's not super clear yet. How do you simplify a group into a person, that part is not clear enough. Or is it? In 3-1 and 3-2?
