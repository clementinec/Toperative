Originally developed to ensure the productivity of occupants in relation to industrial hygiene\cite{bedford_globe_1934}, thermal comfort is a topic that exhibits linkage to health and well-being as well as the learning capabilities. Its definition remains as `that condition of mind that expresses satisfaction with the thermal environment' as it was in 1966 (Standard 55-1966), but also needs to be `assessed by subjective evaluation' \cite{ansi/ashrae_standard_2017} according to the latest standard published in 2017. This is an interesting change that marks two things: First, the challenge posed by unsatisfactory indoor climate remains 46 years after Fanger's PMV/PPD model appeared to have addressed the long-standing challenge of quantifying thermal comfort through physical parameters\cite{fanger_assessment_1973}. Second, the subjective evaluation of the thermal environment from individual occupant is also crucial to the correct characterization of thermal comfort. 
%Thermal comfort is important.

The need to understand the thermal comfort of the occupants in the urban environment has also been growing during the last few decades. Attempting to ensure social equity while designing urban spaces and addressing the heat-related mortality, metrics including the directly measurable $W/m^2$ (or as later translated to mean radiant temperature - CITE) as well as simulation-based/complicated Physiologically Equivalent Temperature\cite{h._hoppe_new_1992} and Universal Thermal Climate Index (UTCI) became widely used among urban climatologists\cite{hoppe_different_2002}. 

Despite these efforts, ensuring the thermal comfort for all occupants appears to have remained a huge challenge. It is precisely due to the ample amount of research and their application in the building industry that the thermal comfort of occupants go through a two-stage simplification: first, the occupants of different demographics are simplified into a hypothetical person; second, this hypothetical person is then simplified into the combination of a few, or a single a single environmental parameter, or more explicitly, the air temperature within the state-of-the-art building systems. Even when there are multiple environmental parameters included in the building automation control, most of the other parameters are often supplementary while air temperature remains to be the main feedback variable. 


This resulted in not only rapid increase of occupants dissatisfied with the indoor environment during the last decades, but also a growing amount of concentration on improving the indoor thermal comfort. Many researchers uses the concept of performance gap to explain the unpredictability of post-occupancy stage\cite{shi_magnitude_2019,allard_energy_2018}, which can be viewed as an attitude of compromise to the challenge: the occupants and their behaviors are beyond prediction and therefore the regulations and mandates of the thermal comfort should be loosen up. For researchers who are insisting that the behaviors of the occupants can still be modelled and predicted, machine and reinforced learning\cite{peng_using_2018}, artificial neural networks\cite{sugimoto_human_2013} as well as model predictive control \cite{jazizadeh_user-led_2014,brooks_experimental_2014} are common approaches used in identifying the occupants' preference and behaviors.

In the meantime, there have already been many reviews on the thermal comfort of the occupants, the differences between thermal preferences/sensations/perceptions\cite{charles_fangers_2003}, and how using either adaptive \cite{nicol_adaptive_2002,nicol_overview_2010} or personalized thermal comfort models\cite{kim_personal_2018} might be able to solve this long-standing problem. These studies spans across the last two decades and utilizes the states-of-the-art techniques, but has yet to create a satisfying solution for fatiguing battle with the indoor environment\cite{noauthor_occupational_nodate}.

We hope to contribute to the understanding and characterization of thermal comfort from a more bottom-up perspective in this paper. Unlike previous researchers who focused more on providing a solution that easily quantifies the thermal comfort of the occupant, we want to examine the existing comfort metrics, their underlying relationship with the occupants, and the simplifications or assumptions that are currently used in conjunction of these models' deployment in existing systems. In order to do so, we have examined both the existing metrics of thermal comfort for the indoor and outdoor environment, and how some unintended simplifications took place during the process of these metrics' proliferation. We also documented and reviewed some of the latest efforts to address this from either a top-down crowd-modeling approach\cite{salamone_integrated_2018} or calibrating existing control algorithms with actual occupant votes\cite{gao_using_2010} or feedback from wearable sensors\cite{abdallah_moatassem_sensing_2017}. We conclude that it is extremely crucial to include the actual response - direct or indirect  - from the occupants into the control logic of the building automation system with additional energy and comfort benefits. 

However, this does not mean that we understood the thermal comfort accurately. 
%Indoor and outdoor metric
Extremely well-conditioned systems requires significant financial investment - both the capital and the operational costs. 
%Efforts beyond environmental parameters 

Energy consumption of buildings to ensure comfort delivery gradually increases, casting even larger pressure on providing improved thermal comfort with smaller energy budget. Under the premises of increasing demand of thermal comfort, designing systems and buildings that provide better comfort without excessive energy consumption becomes more important.

This can obviously be investigated by proposing alternative solutions that provides agreeable thermal comfort at smaller energy costs spent on heating/cooling. However, recent studies that links improved PMV/PPD values with improved designs have showed that the resulting satisfaction of the occupants and the higher PMV/PPD values do not always coincide. Existing studies attempts to point these results to the individual differences between occupants, where metabolic rates and various individual thermal preferences were used as potential explanations for these results.

We want to take an alternative route to tackle this problem in this paper by examining the metrics used when characterizing thermal comfort - more specifically on the assumptions and simplifications used in the conventional methods. Specifically, we would like to provide answers to these following questions: 

\begin{itemize}
\item What has been decided as necessary inputs to characterize occupant thermal comfort?
\item What are we currently (actually) measuring instead?
\item How are we justifying not measuring all of the required inputs?
\item How will these simplifications affect our understanding of the occupants’ comfort and behavior?
\end{itemize}

It's also important to emphasize that this paper specifically address the negligence of the occupants and their physiological responses within the current status-quo practices and standards, and how it's often grossly interpreted into overtly simplified metrics such as operative temperatures or even air temperatures.  

Majority of the methods we as designers and engineers are currently using to approach thermal comfort simplifies a group of occupants into a hypothetical average person, whose thermal comfort is further simplified into a combination of a few or a single environmental parameter. To better understand the function and meaning of these simplifications and the assumptions they were based on, we propose to examine both of these paths in this paper: Regarding the simplification from a hypothetical average occupant being simplified into a couple or even a single environmental parameter, we primarily focus on the required inputs, prevailing underlying assumptions 
%Add a subsection on the physiological responses of the occupants, which is often overlooked. Maybe also on the heat budget model of the human body. Also reference Gagge during this process. Also because we want to talk about this, it really is necessary for the models to come before it rather than after. 
as well as apparatus and tools used to assess a thermal environment. 