% !TEX root = all.tex

\subsubsection{Non-measurable metrics}
    A significant portion of the metrics used in the outdoor environment when assessing thermal comfort are not directly measurable
    %NOTE: THIS IS BEYOND OCCUPANT EVEN, as there's NO WAY to validate the simulated results other than their comparative differences, emphasize on how many different variables are involved during this process. 
    Among these variables, the physiologically equivalent temperature (PET) and the universal thermal climate index (UTCI) are becoming increasingly popular.
    \paragraph{PET} is first proposed by Hoppe who pointed out the necessity of a universal index describing the well-being of the occupant for both the indoor and outdoor environment. 
    
    benefits and caveats of PET
    

    The physiological equivalent temperature (PET) was proposed by Hoppe\cite{h._hoppe_new_1992}   
    \paragraph{UTCI}, or universal thermal climate index is a concept specifically developed by urban climatologists to address the challenges posed by the ongoing/upcoming challenges observed in the urban environment.
    %How it was first proposed
    
    Its application among the existing studies primarily sits within the urban climate studies and simulations.
    
    It's important to point out that all these metrics are non-measurable and unverifiable on their own. Unlike vote-like systems such as PMV that the results can directly be compared with the actual mean vote among the occupants, these variables cannot be compared with directly measured values without making assumptions about what is the best proxy of comfort. 
    Attempts to verify the simulated results often ends up  
    A common trait between these variables is that none of them can be actually measured or calibrated. 
\subsubsection{Measurable metrics}
        W/m2
        $W/m^2$ is the surface-area-averaged incoming radiation at any given time. It can be directly measured by radiant flux sensors - often a thermopile (Jones, 1985). Thermopiles are capable of measuring the voltage difference generated between two sets of 
        WBGT -> wet bulb globe temperature
        MRT -> Application of globe thermometers
