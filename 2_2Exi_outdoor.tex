% !TEX root = all.tex
With respect to how the thermal comfort can be characterized within a thermal environment, the existing literature often uses indoor/outdoor to categorize them(Rupp, 2015). This is a valid classification method when considering the environments to be either built (indoor) or natural (outdoor), which is also a natural result of the indoor and outdoor research communities being independent of each other: the indoor community focuses on ensuring the built environment satisfiable, following often either the ISO 7730 \cite{iso_iso_2005} or the ASHRAE Standard 55 \cite{ansi/ashrae_standard_2017} while the outdoor community deals with a wider range of environmental parameters and are often more concentrated on heat stresses ISO 7243 (7243, 2017). When it comes to the actual method of evaluation, we believe we can also categorize these thermal comfort metrics into whether they're directly measurable or not. Understanding whether resulting indices are directly linked to how the outdoor thermal comfort can be characterized. Chen and Ng provided a very comprehensive review on the existing models and methods in characterizin gthe outdoor thermal comfort, specifically on how the different metrics are developed. However, as their classifiers are the physical, physiological, psychological and social/behavioral aspects of these metrics, we want to provide an alternative aspect in understanding these metrics. 

\subsubsection{Non-measurable metrics}
    A significant portion of the metrics used in the outdoor environment when assessing thermal comfort are not directly measurable
    %NOTE: THIS IS BEYOND OCCUPANT EVEN, as there's NO WAY to validate the simulated results other than their comparative differences, emphasize on how many different variables are involved during this process. 
    Among these variables, the physiologically equivalent temperature (PET) and the universal thermal climate index (UTCI) are becoming increasingly popular.
    
    The physiological equivalent temperature (PET) is first proposed by Hoppe who pointed out the necessity of a universal index describing the well-being of the occupant for both the indoor and outdoor environment. PET was proposed by Hoppe\cite{h._hoppe_new_1992} as a metric that quantifies the surrounding environment's effect on occupants. It is a static metric that characterize the ocucpant's thermal sensation in an outdoor environment but can also be applied within the indoor environment. This metric promises the output of the overall thermal comfort condition as a temperature-like metric, which cannot be verified with any kind of measurement. Common simplifications to verify the PET includes using the operative while acknowledging the limitations of this verification in the meantime. 

    A similar metric, the universal thermal climate index \cite{blazejczyk_comparison_2012} is a concept specifically developed by urban climatologists to address the challenges posed by the ongoing/upcoming challenges observed in the urban environment. This is a metric based on multi-node thermoregulation modeling, and is commonly considered more comprehensive than PET. However, since this model requires even more information specific to the occupants, it is more challenging to measure all necessary inputs to this model, which have led to its wider application in urban modeling - and validation only within the scope of measured air temperature, mean radiant temperature (or operative temperature) as well as other environmental parameters such as the air velocity, relative humidity, etc\cite{nikolopoulou_thermal_2001}. 
    %How it was first proposed

    While these metrics are non-measurable and unverifiable on their own, they all produce a somewhat temperature-like output. Unlike vote-like systems such as PMV/PPD whose results can directly be compared with the actual mean vote among the occupants, these variables cannot be compared with directly measured values without making assumptions about what is the best proxy of comfort. The most commonly used measurable variable remains to be operative temperature for these metrics, whose limitations we have already pointed out within previous section. In the meantime, it is also important to point out the caveat of the air velocity we identified inherent to the operative temperature is even less desirable for its application in the outdoor environment, and should be better characterized to be used as the method of validation. 
\subsubsection{Measurable metrics}
    For the measuarble metrics of thermal comfort within the outdoor environment, there are mainly two categories: metrics that are measurable by globe thermometers and metrics that are measurable with net radiometers. Neither group consideres the individual occupants, and provides only a range of acceptable instead of agreeable thermal environment - which is helpful for more extreme outdoor environments.  

    Globe thermometers are one of the most common apparatus when measuring mean radiant temperatures. When in thermal equilibrium, their readings reflect both the convective and radiative heat transfer between the globe and the surrounding environment. By correcting the convective heat transfer with known air temperature and air velocity, it is possible to approximate the mean radiant temperature from globe thermometer readings\cite{standardization_iso7726_2001}. Many studies have already pointed to the limitations of globe thermometers, citing their limitations of settling time, tradeoffs between the settling time \cite{graves_globe_1974} and size of the globe \cite{dear_ping-pong_1988} and the general lack of accuracy of globe thermometers \cite{aparicio_globe_2016}. However, the ease of use of globe thermometers and their comparative lower price have halped them to maintain their popularity in existing research. While direct use of globe temperature as an indicator of thermal comfort is rare for research in the outdoor environment, indices including the wet-bulb globe temperature \cite{james_h._botsford_wet_1971} have gained popularity among urban climatologists. 

    %W/m2
    Contrast to the globe thermometers, net radiometers uses three sets of up and down radiant flux sensors to capture the integral radiation from all directions within a measured location. Although often later converted into mean radiant temperatures, the actual measured value is the radiant flux, or $W/m^2$. 
    $W/m^2$ is the surface-area-averaged incoming radiation at any given time. It can be directly measured by radiant flux sensors - often a thermopile (Jones, 1985). Thermopiles are capable of generating various votltages, thus by measuring the voltage differences generated between overall incoming radiant flux and the surrounding environment, and can therefore be used to measure the radiant flux within measured field of view. This metric is not only independent of the occupant, but also of the air temperature, offering only the characterization of the radiant environment. As researchers have demonstrated that the resulting mean radiant temperature calculated from the net radiometers could reach up to 65$\degree C$ and creates threats on the human health when the air temperature is no more than 30 $\degree C$, incoming radiant fluxes captured (or $W/m^2$) remains a helpful tool in measuring the outdoor environment. 