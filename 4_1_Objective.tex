% !TEX root = all.tex
%This entire chapter should focus on what has been done. 
Regarding the latest research focusing on addressing the two-step simplification of occupants into environmental parameters, we can categorize them based on their means of planned intervention and its relationship to the occupants into objective and subjective. For the objective interventions, occupants' physiological responses are directly measured through different sensing techniques, and calibrated by the actual votes of the occupants. 

\subsubsection{Direct physiological parameters from wearable sensors}
Measuring the physiological response of the human body has been historically challenging. Building on the principle that physiological responses can be correlated with thermal discomfort \cite{huizenga_skin_2004,}. Monitoring the physiological signals of the human body allows the detection of discomfort signals - and when these signals are absent, it is possible to hypothesize the occupants are comfortable (Takada, 2013; Bicego, 2007). 


Among other measurable responses, the skin temperature of the human body is more common in existing research. To accurately measure the physiological responses of the human body, traditional sensing techniques are often intrusive: sensors that needs to be strapped onto the body \cite{mccarthy_validation_2016}, or needing to be ingested (e.g. core-temperature-measuring radio pill) effectively make it impossible to measure real-time responses for multiple multiples in a real office environment. 

Development in sensing technology has helped researchers to come up with solutions to the comfort conundrum. 

Studies on how sensitive different locations of the human skin are sensitive to changes of the thermal environment have pointed to a few specific locations including the wrist temperature, which later became a major research concentration for researchers\cite{choi_cobi:_2010}. 

%Research regarding the skin temperatures
A particularly popular measurement technique to measure the physiological signal of the occupants is measuring the skin temperature, where different sensors are often placed or strapped onto the test subjects\cite{}
%Research regarding the wrist temperature 

%Research regarding the IoT sensors


? Occupancy, infrared-based occupancy sensing? Can we suggest that is also measuring 
\subsubsection{Thermal Imagery and other direct measurement techniques}
	A relatively less-used method, thermal imagery (or infrared thermography) has also been used in characterizing the thermal responses of the occupants. Its earlier examples can be traced back to as early as 1970s (Cena,1981), the development of the technology has allowed portable IR cameras to become popular tools in energy audit for buildings\cite{lucchi_applications_2018}. As the resolution of the IR cameras gradually improves, its usage in energy audits widens, gradually extending to monitoring not only thermal bridges or subsurface defects, but also the thermo-physical reactions from the occupants to the surrounding environments. %What's the Microsoft Research? 

	Estimating the thermal comfort of individual occupants was also achieved by the usage of multiple non-contact infrared sensors in some studies. Ghahramani et al., for example, demonstrated that it was possible to use 4 points (forehead, hear, bridge of nose and cheekbone)\cite{ghahramani_infrared_2016}. 
% \subsubsection{Globe Thermometer}
%             It’s first introduction (indoor)
%             Limitations recognized - and simplification introduced
%             One MRT per room.
%             Proliferation/increase of its usage (citations? How do we quantify its virus-like applications within both the indoor and outdoor environment)
%             Linkage to Top - (Do we want to discuss the relationship between Top and the convection here?)
%             prior.
            
% \subsubsection{Net Radiometer}
%             Outdoor introduction
%             Expenses, immobilized
%             Marty as being mobilized - CITATIONS
%             Less commonly used since introduction of globe thermometers
            
% \subsubsection{Operative temperature sensors}
%                 Globe thermometers >10
% \subsubsection{Air Temperature Sensors}
%                 Less documented but more often used in real systems.
%                 Air = MRT as a simplification built beyond One MRT per room.
  