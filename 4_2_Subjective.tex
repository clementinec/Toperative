We categorize studies that uses the subjective votes without direct physiological measurements as subjective. Occupants are no longer providing involuntary feedback but rather intentional, voluntary responses as signals collected by the control system or algorithm. 

%Preferences, sensations, etc. 

%But can we consider preferences also as objective? Due to the individual differences, or variations of the metabolic rates?




%Regarding the thermal comfort zone - anlaytical tool?
The latest standard published on determining the indoor occupants' thermal comfort is the ANSI/ASHRAE Standard 55-2017, which supersedes ANSI/ASHRAE Standard 55-2013, which is partially in agreement with ISO 7730, published as the ASRAE 55 Thermal Comfort Tool by the Center of Built Environment at Berkeley in 2017(Insert web citation). 

%What is comfort zone - it's verbal definition and graphical representation
As briefly introduced previously, the comfort zone is defined as combinations of air temperature, mean radiant temperature $\bar t_r$  and humidity that are predicted to be an acceptable thermal environment at particular values of air speed, metabolic rate and clothing insulation $I_{cl}$(ASHRAE Standard 2017). It's also more commonly understood as two overlapping zones represented on psychrometric chart, where the air conditions are solved from a PMV of -0.5 to 0.5. This graphical approach assumes the rest of the environmental parameters (MRT, air velocity) and personal parameters (metabolic rate and clothing factors) as constants, and is the most widely accepted representation of the concept. There are actually two different ways of obtaining these boundary lines at PMV of -0.5 and 0.5. %Need to present it?

%How we determine the comfort zone - three ways in ASHRAE.
To determine the boundary of the comfort zone, both ISO 7730 and ASHRAE Standard 55 provide guidelines on how to do the actual calculation. There are currently three methods outlined in ASHRAE Standard 55, which applies to diffrent ranges of average air speed. With air speed lower than 0.2 m/s and a humidity ratio smaller than 0.012 kg$\cdot H_2O/kg$ dry air, the graphic comfort zone method should be used, where the operative temperature can be determined by linear interpolation the upper and lower operative temperature limit with a given clothing insulation. Alternatively, the comfort zone's boundaries can also be determined by using the Analytical Comfort Zone Method. This method can be applied to metabolic rate between 1.0 and 2.0, and clothing factor between 0 to 1.5 for all humidity ratios. This method incorporates the PMV calculation method used by ISO 7730.