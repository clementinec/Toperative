% !TEX root = all.tex

We categorize studies that uses the subjective votes without direct physiological measurements as subjective. Occupants are no longer providing involuntary feedback but rather intentional, voluntary responses as signals collected by the control system or algorithm. These subjective responses - usually interms of thermal sensation votes (TSV), or actual sensation votes (ASV). Occupants are often asked to vote their subjective evaluation of the environment with a scale of -3 to +3, with 0 pointing to thermal neutrality. 

Without using any intrusive direct measurement techniques, subjective evaluations of the thermal environments are often used as the ground truth when training models with various physiological and environmental inputs(Bermejo, 2012. 

There are also some inherent weakness of collecting subjective feedbacks from the feedback, particularly due the ambiguity of what occupants are actually voting on. Corresponding to the results calculated by PMV model, occupants are meant to vote on how warm/cool they are feeling. Yet thermal sensation is conceptually different from thermal satisfaction, thermal acceptability or thermal preference\cite{charles_fangers_2003}. Although it is obvious that there will be overlap between the different responses, findings from various researh have already pointed out to the discrepancies between these concepts: Paciuk and Becker (2002) found that almost half of the occupants in a naturally-ventilated homes voting $\pm1$ reported they were not comfortable (five-point thermal comfort scale) when they should be considered comfortable according to the ASHRAE mandate. Meanwhile, when residents voted comfortable in an air-conditioned room, their thermal sensation responses sat between -3 to +1. For occupants voting on the extreme - i.e. ASHRAE-stipulated discomfort - thermal sensation scales, Schiller (1990) reported that 27\% to 39\% were moderately or very comfortable (six-point thermal comfort scale). In relation to the thermal preferences, Brager et al. (1994) found that 11 to 36\% of the occupants voting thermal neutral would in fact prefer to feel warmer or cooler. The same study also reported taht for the extreme sesation scales, 3-50\% would prefer no change in temperature, and up to 66\% of them reported they were comfortable (six-point thermal comfort scale). Overall, these findings points to the inconsistencies between the thermal sensation and other measures of thermal comfort evaluations.

To avoid the bias of using only the subjective votes of the occupants, but still capture the individual differences between the occupants, their subjective responses alongside their interactions with the personal thermal comfort devices are used to characterize their respective thermal comfort. The Center of Built Environment in Berkeley is among the strongest promoter of this paradigm. Kim et al. recognized this patter in one of their latest publications, recognizing this approach called a personal comfort model, with five key traits: individual persons are considered instead of larger groups of people, collect direct feedback from individuals are used to train models, is data-driven and can therefore be analyzed with different models and can be further adapted when new data is introduced. Recognizing the recent trend of intelligent comfort management (Talon, 2015, Solutions are changing the occupant experience, navigant consulting). This falls into behavioral adaptation, which, alongside outdoor climate, are the two major categories of reasons that explains the lack of PMV's accuracy in predicting occupants' thermal comfort\cite{charles_fangers_2003}. 

Many personal thermal comfort models have been developed so far, where actual feedback data from the occupants are used to train the models to predict the individual states of thermal sensation, and are often coupled with either statistical/probabilistic methods \cite{daum_personalized_2011} or machine learning models. Many of these studies uses environmental parameters as model inputs, and physiological measurements \cite{ghahramani_online_2015} as supplementary data as well as occupants feedback as thermal sensation ground truth\cite{liu_personal_2019}. Most of these physiological measurements are collected through wearable sensors that were designed to minimize the intrusion of the occupants - larger devices such as the face-mounted frame used by Ghahramani et al. \cite{ghahramani_infrared_2016} are rarer, while smaller fitness trackers such as fitbit or Microsoft band (Li, 2017, Personal) are much more popular due to the lower costs of prototype development. Infrared thermography are also used in some cases \cite{lu_thermal_2019} but are often limited with the field of view and resolution of the incoming infrared feed. 

These efforts are also categorized as a shift from centralized thermal comfort to personzlied thermal comfort in some recent research where the individualized differences are emphasized and magnified \cite{wang_individual_2018}. However, despite recognizing the need to ensure the personal thermal comfort of the occupants through individual modeling, the negotiation between the predicted thermal comfort of the occupants and how their individual `comfort' may overlap remains a topic that has not yet been investigated. 

%Preferences, sensations, etc. VARIATIONS
It should also be noted that while the terms thermal comfort, thermal sensation and thermal preference are often used interchangeably in some publications, their overlaps are limited, and may therefore result in over or under estimation of the state of thermal comfort among occupants. Existing literature has ample evidence on how the three concept may differ. 


%But can we consider preferences also as objective? Due to the individual differences, or variations of the metabolic rates?


%What are some of the machine learning efforts nonetheless? What have we already done within this scope?
